\documentclass[a4paper,12pt, twoside]{memoir}   	% use "amsart" instead of "article" for AMSLaTeX format
\usepackage{geometry}                		% See geometry.pdf to learn the layout options. There are lots.
%% \geometry{letterpaper}                   		% ... or a4paper or a5paper or ...
\geometry{vmargin=3cm}%% hmargin=2cm
\usepackage{graphicx}
\usepackage{color}
\usepackage[dvipsnames]{xcolor}
\usepackage{enumerate}
\usepackage{amssymb}
\usepackage{footnote}
\usepackage{enumerate}
\usepackage{eucal}
\usepackage{amsmath}
\usepackage{braket}
\usepackage[utf8]{inputenc}
\usepackage[english]{babel}
\usepackage{hyperref}
\usepackage{xspace}
\usepackage{pdfpages}
\usepackage{subcaption}
\usepackage[margin=2cm]{caption}

\captionsetup[figure]{font=footnotesize}
\captionsetup[subfigure]{font=footnotesize}

\hypersetup{
  colorlinks,
  citecolor=red,
  filecolor=red,
  linkcolor=red,
  urlcolor=red
}

\setcounter{tocdepth}{3}
\setcounter{secnumdepth}{3}

\newcommand{\zeronu}{0\nu\beta\beta}
\newcommand{\twonu}{2\nu\beta\beta}
\newcommand{\Tl}{$^{208}$Tl}
\newcommand{\Co}{$^{60}$Co}
\newcommand{\Pb}{$^{208}$Pb}
\newcommand{\Bi}{$^{214}$Bi}
\newcommand{\Qbb}{Q_{\beta\beta}}
\newcommand{\Bckunit}{\text{counts.keV}^{-1}\text{.kg}^{-1}\text{.y}^{-1}}
\newcommand{\Tbeta}{T_{1/2}^{0\nu}}
\newcommand{\mbb}{m_{\beta\beta}}

\newcommand\myshade{85}
\colorlet{mylinkcolor}{violet}
\colorlet{mycitecolor}{YellowOrange}
\colorlet{myurlcolor}{Aquamarine}
\usepackage{hyperref} %backref
\hypersetup{
  linkcolor  = mylinkcolor!\myshade!black,
  citecolor  = mycitecolor!\myshade!black,
  urlcolor   = myurlcolor!\myshade!black,
  colorlinks = true,
}

\newenvironment{itemize*}%
               {\begin{itemize}%
                   \setlength{\itemsep}{0pt}%
                   %\setlength{\parskip}{0pt}
               }%
               {\end{itemize}}




               \begin{document}

               \chapter{Characterisation of the calorimeter time resolution}
               \section{Measurement of the time resolution with a \Co\ source}
               \begin{itemize}
               \item est-ce qu'avec la digitisation du pulse on pourrait distinguer un signal électron d'un signal gamma, voire pour un gamma avoir une indication même grossière de la profondeur dans le scintillateur où il a interagi?
               \item Corrélation angulaire des deux gammas de 1.17 et 1.33 MeV1+1/8*cos**2+1.24*cos**4
               \item est-ce que la valeur fittée pour la calib en énergie avec le Cobalt (0.92) dépend de la géométrie (distance de l'OM à la source, angle du photon...)
               \item si le 208Tl est la cause de la queue à haute énergie, tu pourrais (pour les blocs bien calibrés) faire le biplot (Emin, Emax) pour les événements d'énergie supérieure à 1,5 MeV (que tu n'as pas dans les simulations.)
               \item à faire : regarder la différence des moyennes des distrib delta t pour savoir d'où ça vient.
               \end{itemize}


               \chapter{Sensitivity study}
               \begin{itemize}
               \item Modification du Z de la source (par ex Nd->Se) peut avoir un faible effet sur le internal bkg: petit changement de Z du milieu qui peut modifier la probabilité d'effet Compton ou Möller
               \item Pour retrouver le rapport entre Radon et 214Bi
                 Radon (en supposant que seule la première rangée de cellule contribue) : 0,15 mBq/m3 *15,3 m3 *1/9 = 0,255 mBq
                 214Bi : 0,010 mBq/kg * 6,23  kg = 0,063 mBq
                 soit un rapport 3 entre les 2 (bon tu as sans doute un peu moins car dans le cas du Radon on voit plus souvent au moins une cellule retardée donc on peut le rejeter, alors que dans le cas du Bi214 l'alpha peut être complètement absorbée dans la feuille source.
               \item  le fait que le champ soit moins intense, s'il restait homogène, ne changerait pas le fit en hélice des traces, simplement à plus haute énergie on aurait du mal à distinguer les électrons des positrons.
                 Le champ homogène complique les choses car la trajectoire n'est plus une hélice.
                 Sans compter qu'avec les pertes en énergie dans le tracker, l'hélice n'est sans doute pas le modèle idéal pour fitter les trajectoires des particules.
               \item If I understand correctly you are speaking about NEMO-3 Mo-100 data.
                 The external background will increase by 0.73 events if
                 instead of taking the [2.8,3.2] ROI of NEMO-3, we take a [2.7,3.15] energy range.
                 At the moment when the paper have been written we did not worry much about
                 optimization of the boundaries of the region of the interest, since the 0-nu limit
                 setting was performed with Collie using the full spectrum.
                 The ROI was not used for the limit setting, it was needed just to demonstrate that
                 we observe roughly the expected number of events without any significant excess.
               \item bdf externe neutrons : il s'agit de l'interaction de neutrons thermiques dans le blindage en fer, donc de neutrons qui se sont préalablement thermalisés (notamment dans les scintillateurs plastiques)
               \item Le fait qu'on n'ait plus d'evts 2n2b au-delà de 2,85 MeV est dû au fait qu'on simule au-delà de 2 MeV. Sans-doute aurait-il fallu simuler de la 2n2b au-delà de 2,3 ou 2,5 MeV pour avoir quelques evts après. Cela dit je ne pense pas que cela affecte ton étude de sensibilité. (Cela aurait été le cas si la fenêtre optimisée démarrait plus tôt)
               \item Remarque : la borne supérieure de la ROI pourrait avoir une influence si on considérait certains bruits de fond comme les neutrons.
               \item C'est un détail mais réfléchir au fait que le noyau fils recule, a donc une impulsion (très faible mais d'ailleurs c'est à cause de cela que les deux électrons ne sont pas émis dos à dos dans la 0nu), donc peut-être que le pic n'est pas infiniment fin.
               \item Juste en remarque : la coupure sur Pint permet de rejeter seulement 10\% du 208Tl, alors qu'elle permet de rejeter 50\% du Radon.
               \item En remarque, il y aurait sans doute aussi des variations de la direction du champ
                 Il faudra se préparer à la question: pourquoi on perd un facteur ~2 sur la sensibilité sur la demi-vie par rapport aux études précédentes?
                 réponse :
                 ->likelihood, pas de radon, act. différentes
                 ->regarder mes résultats sans radon
                 ->efficacité 0nu surestimée (ancien soft)
                 \item Remarque: l'utilisation de l'énergie individuelle permet de rejeter du 208Tl par exemple.

Cependant vu que tu as déjà très peu de bruit de fond, est-ce que tu vas gagner beaucoup? Peut-être considérer aussi les evts à 2 électrons d'énergie un peu plus faible, ce -> il serait poss d'avoir un Emin de la ROI plus petit si on arrive à couper Tl avec coupure sur Eindiv
                   \item Remarque: pour le processus de neutrino de Majorana léger, je ne suis pas sûr qu'on gagne beaucoup avec une analyse multivariée
(Par exemple entre 2n2b et 0n2b les variables cinématiques sont assez similaires, et le 2n2b est le bruit de fond dominant.)
               \end{itemize}

               \chapter{Tl analysis}
               \begin{itemize}
               \item Pour la distrib Pint pour la 0nu : pourquoi tu as un pic à zéro, même dans le cas sigma L = 0.038 ns (Est-ce que ce pic est moins important si tu sélectionnes des événements dans le RoI? Sinon diffusion multiple dans le tracker, retrodiffusion à la surface d'un scintillateur...). Pourquoi tu as un pic à zéro, même dans le cas sigma L = 0.038 (Est-ce que ce pic est moins important si tu sélectionnes des événements dans le RoI? Sinon diffusion multiple dans le tracker, retrodiffusion à la surface d'un scintillateur...)
               \end{itemize}

               \end{document}

               \chapter{Detector}

               \begin{itemize}
               \item due to difficulties with the mechanical design and degradation of $\Delta E/E$ as a result of the energy loss of electrons in the entrance window of the liquid scintillator container these options were not considered for the final design.
               \item La granularité di calo peut jouer si on empile un gamma et un électron.
                 Cela avait été discuté quand on avait réfléchi à un calorimètre en barre : moins coûteux mais à priori moins performant.
                 \item Remarque pour tag du bdf Tl dans le canal 1eng: dans la très grande majorité des cas, c'est au moins 2 gammas mais on peut louper la détection de 1 gamma, ce qui donne des événements à 1 électron et 1 gamma.
               \end{itemize}


               \chapter{Commissioning}
               \begin{itemize}
               \item Une remarque (à réfléchir pour plus tard).
                 On pourrait te dire que le sigma obtenu est dans tous les cas négligeable devant la précision de la mesure du temps 100, 200 ou 400 ps, donc que cette optimisation n'est pas très utile.
               \end{itemize}
