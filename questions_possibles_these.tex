\documentclass[a4paper,12pt, twoside]{memoir}   	% use "amsart" instead of "article" for AMSLaTeX format
\usepackage{geometry}                		% See geometry.pdf to learn the layout options. There are lots.
%% \geometry{letterpaper}                   		% ... or a4paper or a5paper or ...
\geometry{vmargin=3cm}%% hmargin=2cm
\usepackage{graphicx}
\usepackage{color}
\usepackage[dvipsnames]{xcolor}
\usepackage{enumerate}
\usepackage{amssymb}
\usepackage{footnote}
\usepackage{enumerate}
\usepackage{eucal}
\usepackage{amsmath}
\usepackage{braket}
\usepackage[utf8]{inputenc}
\usepackage[english]{babel}
\usepackage{hyperref}
\usepackage{xspace}
\usepackage{pdfpages}
\usepackage{subcaption}
\usepackage[margin=2cm]{caption}

\captionsetup[figure]{font=footnotesize}
\captionsetup[subfigure]{font=footnotesize}

\hypersetup{
    colorlinks,
    citecolor=red,
    filecolor=red,
    linkcolor=red,
    urlcolor=red
}

\setcounter{tocdepth}{3}
\setcounter{secnumdepth}{3}

\newcommand{\zeronu}{0\nu\beta\beta}
\newcommand{\twonu}{2\nu\beta\beta}
\newcommand{\Tl}{$^{208}$Tl}
\newcommand{\Co}{$^{60}$Co}
\newcommand{\Pb}{$^{208}$Pb}
\newcommand{\Bi}{$^{214}$Bi}
\newcommand{\Qbb}{Q_{\beta\beta}}
\newcommand{\Bckunit}{\text{counts.keV}^{-1}\text{.kg}^{-1}\text{.y}^{-1}}
\newcommand{\Tbeta}{T_{1/2}^{0\nu}}
\newcommand{\mbb}{m_{\beta\beta}}

\newcommand\myshade{85}
\colorlet{mylinkcolor}{violet}
\colorlet{mycitecolor}{YellowOrange}
\colorlet{myurlcolor}{Aquamarine}
\usepackage{hyperref} %backref
\hypersetup{
  linkcolor  = mylinkcolor!\myshade!black,
  citecolor  = mycitecolor!\myshade!black,
  urlcolor   = myurlcolor!\myshade!black,
  colorlinks = true,
}

\newenvironment{itemize*}%
  {\begin{itemize}%
    \setlength{\itemsep}{0pt}%
    %\setlength{\parskip}{0pt}
  }%
  {\end{itemize}}




\begin{document}

\chapter{Characterisation of the calorimeter time resolution}
\section{Measurement of the time resolution with a \Co\ source}
\begin{itemize}
\item est-ce qu'avec la digitisation du pulse on pourrait distinguer un signal électron d'un signal gamma, voire pour un gamma avoir une indication même grossière de la profondeur dans le scintillateur où il a interagi?
\item Corrélation angulaire des deux gammas de 1.17 et 1.33 MeV1+1/8*cos**2+1.24*cos**4
\end{itemize}


\chapter{Sensitivity study}
\begin{itemize}
\item Modification du Z de la source (par ex Nd->Se) peut avoir un faible effet sur le internal bkg: petit changement de Z du milieu qui peut modifier la probabilité d'effet Compton ou Möller
\item Pour retrouver le rapport entre Radon et 214Bi
Radon (en supposant que seule la première rangée de cellule contribue) : 0,15 mBq/m3 *15,3 m3 *1/9 = 0,255 mBq
214Bi : 0,010 mBq/kg * 6,23  kg = 0,063 mBq
soit un rapport 3 entre les 2 (bon tu as sans doute un peu moins car dans le cas du Radon on voit plus souvent au moins une cellule retardée donc on peut le rejeter, alors que dans le cas du Bi214 l'alpha peut être complètement absorbée dans la feuille source.
\item  le fait que le champ soit moins intense, s'il restait homogène, ne changerait pas le fit en hélice des traces, simplement à plus haute énergie on aurait du mal à distinguer les électrons des positrons.
Le champ homogène complique les choses car la trajectoire n'est plus une hélice.
Sans compter qu'avec les pertes en énergie dans le tracker, l'hélice n'est sans doute pas le modèle idéal pour fitter les trajectoires des particules.
\item If I understand correctly you are speaking about NEMO-3 Mo-100 data.
The external background will increase by 0.73 events if
instead of taking the [2.8,3.2] ROI of NEMO-3, we take a [2.7,3.15] energy range.
At the moment when the paper have been written we did not worry much about
optimization of the boundaries of the region of the interest, since the 0-nu limit
setting was performed with Collie using the full spectrum.
The ROI was not used for the limit setting, it was needed just to demonstrate that
we observe roughly the expected number of events without any significant excess.
\end{itemize}

\end{document}
