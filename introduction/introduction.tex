\chapter*{Introduction}
\label{ch:intro}
\addcontentsline{toc}{chapter}{Introduction}

It is always interesting to take a historical approach when talking about a scientific discovery. This allows us to put into perspective knowledge that is now considered to have been acquired.

The Standard Model of Elementary Particle Physics attempts to describe the world around us on scales that were inconceivable two centuries ago.
A little over a hundred years ago, Henri Becquerel discovered what we today call radioactivity, with the observation of $\beta$ decay.
This historical discovery was nevertheless accompanied by profound questioning, since the $\beta$ particle emitted during this decay, which turned out to be an electron, only carries away part of the available energy for the reaction.
This observation was contrary to the first principle of thermodynamics on the energy conservation, and some scientists postulated that this fundamental law was being violated.
It took 35 years for an eminent scientist by the name of Wolfgang Pauli to propose the existence of the \emph{neutrino} ($\nu$) - for small neutron in Italian - as a solution to the problem of missing energy.
Three years later Enrico Fermi laid the foundations for the first mathematical formulation of what is today the Lagrangian of weak interaction.
It was another 25 years, 60 years after the discovery of $\beta$ radioactivity, before the neutrino was experimentally observed by Clyde Cowan and Frederick Reines.
The neutrino adventure had only just begun.

Why is this particle, although abundantly produced in the sun in the atmosphere and in the earth, so difficult to detect?
It is because it interacts very little with the matter - electrons and quarks - that constitutes us, being sensitive only to the weak interaction (of short range), and to the gravitational force (very weakly since the mass of the neutrino is extremely low, so much that it was believed massless for a long time).

In the current model of particle physics, neutrinos are actually described as massless.
It was Bruno Pontecorvo who proposed in 1957 that neutrinos could oscillate between their different mass states, based on the already known model of oscillation of neutral kaons.
To be valid, this model then presupposed that neutrinos had a non-zero mass.
It was the SuperKamiokande experiment that first observed this phenomenon in 1998, demonstrating that at least two of the three neutrino flavours have a non-zero mass.
The Standard Model of particle physics is then no longer sufficient to account for this particle properties, opening the way to physics beyond the Standard Model.

It now remains to be discovered how this particle acquires its mass.
Indeed, having a neutral charge under the three fundamental interactions described by the Standard Model, two mass generation mechanisms are foreseeable.
The first is to assume that, like all other fermions, the neutrino obtains its mass through the Higgs mechanism, leading irremediably to the assumption of the existence of a sterile neutrino.
The second, proposed by Ettore Majorana, assumes that the neutrino is its own antiparticle, giving the neutrino its mass with the addition in the Lagrangian of the Majorana mass term.
If this assertion is the one that applies to neutrinos, then a disintegration, prohibited in the Standard Model, is possible.
It is called \emph{neutrinoless double beta decay} ($\zeronu$), to contrast with the \emph{two neutrinos double beta decay} ($\twonu$) allowed by the Standard Model and already observed for several isotopes.
In the former disintegration, two simple $\beta$ decays take place simultaneously in the same nucleus, in which the two neutrinos are absorbed, allowing the total energy of the reaction to be distributed between the two exiting electrons.
For reasons that are detailed in the first chapter of this manuscript, which deals with the phenomenology of the neutrino, this disintegration is only possible if the neutrino is a Majorana particle.

Several experiments, also described in the first chapter, are dedicated to the search for this disintegration which, if it exists, is expected to be extremely rare.
%%Its observation would prove the neutrino is a Majorana particle.
The SuperNEMO experiment, on which I conducted my PhD, is one of them.
Successor of the NEMO experiments, it uses a unique combination of technologies, described in detail in the second chapter, allowing to trace the path of the electrons resulting from double $\beta$ disintegrations -with a wire chamber-, and also to measure their energies - with a segmented electromagnetic calorimeter.

These generation of experiments differ from one another in the technology they use, and also in the sensitivity they can achieve in the search for this decay.
Within the framework of this PhD, I carried out a sensitivity study of this experiment presented in the third chapter, determining the influence that several characteristics of the detector can have on it.

All these experiments are designed to observe, should this process exist, an extremely rare physical event.
They are thus constrained to focus on the background which may disturb the measurement and have a non-negligible impact on their sensitivity to this disintegration.
In this perspective, the fourth chapter presents a new technique to identify the events resulting from one of the main background for this experiment, which is the natural disintegration of an isotope from the uranium 238 decay chain, found in the detector's components.
To effectively reject this background, I also study the impact on the sensitivity of the accuracy with which we measure the arrival time of particles in the calorimeter.
To complete this theoretical analysis based on simulations of the detector, the fifth chapter gives an overview of the data taken at Modane with the SuperNEMO calorimeter and of the analysis aiming to characterise the time resolution for a large part of the optical modules.

When I joined the LAL team at Orsay (now IJCLab) as a PhD student, SuperNEMO was already largely built in the Modane underground laboratory.
I had the opportunity to actively participate in the completion of its assembly, as well as in the analyses of the first commissioning data described in the last chapter, thus completing the experimental knowledge acquired during this PhD.
