\chapter{Sensitivity of the SuperNEMO demonstrator to the $\zeronu$}
\label{ch:sensitivity}

But: étudier sensibilité à la 0nu (+effective neutrino masses) du démonstrateur (Se) avec/sans champ qui détermine installation de la bobine ou non\\
Parler du champ non uniforme/attenuation

\section{State of the art}
Citer article SN (2010) + Steven
\section{Signal and backgrounds considered}
Demies vies 2nu à jusifier\\
Activités bkg considérées à justifier\\
Se, Nd, bkg interne (balek externe), avec et sans champs\\
présentation du PID de Steven (peut être à bouger dans Tl selon développement du plan ou généralités)
\section{Event selection}
Quel est le signal qu'on cherche\\
présentation des cuts\\
efficacité des cuts/ signal + bkg
\section{Expected number of background events}


\section{Demonstrator sensitivity}
Résultats $B=0$, avec activités nominales, puis avec activités caca\\
Efficiency spretra\\
Energy spectra
\subsection{avec B}
ROI optimization: avec variation coupure énergie\\
ave/sans B
\subsection{sans B}
avec variation coupure énergie\\
ave/sans B

\section{HyperNEMO}
results for $500$kg.y exposure

\section{Other isotopes}
bam bam le Nd

\section{Conclusion}
Faut arrêter SN\\
Etude plus générale avec bkg externe+lab (reprendre chiffres NEMO3)
+ neutrons (cf NEMO3)
