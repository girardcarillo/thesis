\chapter*{Conclusion}
\addcontentsline{toc}{chapter}{Conclusion}
\label{ch:conclu}

The search for the neutrinoless double beta decay is one of the doorways to physics beyond the Standard Model.
If the neutrino is a Majorana particle, in addition to providing an explanation for the matter/anti-matter asymmetry observed in the universe, the existence of $\zeronu$ decay could explain the fact that neutrinos have a very low mass compared to other fermions, through the see-saw mechanism.

NEMO technology, which has already set limits on the effective neutrino Majorana mass for several isotopes, has given birth to the SuperNEMO detector.
We determined that with $100$~kg of \Se\ this detector based on the unique tracko-calo technology would achieve $\Tbeta~>~5.4\times~10^{25}$~years, corresponding to $\langle\mbb\rangle~=~[0.079-0.15]$~eV, for $5$~years of data acquisition.
The SuperNEMO demonstrator, which is nearing the end of installation at the Modane Underground Laboratory with $6.23$~kg of \Se, will complete its commissioning phase by the end of $2020$ and will take data for slightly more than two years and a half.
With the measurement of SuperNEMO source activities by BiPo-$3$ detector, we also found that the demonstrator should achieve a sensitivity at $\zeronu$ of $\Tbeta~>~3.6\times~10^{24}$~years, corresponding to $\langle\mbb\rangle~<~[0.31-0.59]$~eV.
The $25$ Gauss magnetic field that will be applied in the detector has very limited impact on the sensitivity if optimised topological selections are applied on the event.
Nevertheless, it is necessary to wait for simulations of external background before a more complete study can be carried out and a final conclusion can be drawn on the influence of this magnetic field.
When the demonstrator starts taking data, these activities can be measured more accurately and the sensitivity results will be updated.
In particular, it is conceivable that the contamination of sources in \Bi\ is lower than the upper limit provided by BiPo-$3$.

A way to improve this sensitivity is to reject more efficiently the background coming from \Tl\ isotope decays in the sources.
When this isotope performs a beta decay to an excited level of \Pb\ followed by internal conversion of the $2.615$~MeV metastable level, the event can be rejected by measuring the times of flight of the two detected electrons.
A 6\% improvement in sensitivity was achieved by setting up optimised time-of-flight rejection.

It was demonstrated that this improvement in sensitivity is deeply related to the actual time uncertainty of optical modules, and beyond $\sigma_{t}=200$~ps, no improvement can be reached on the \Tl\ rejection.
A mission was then conducted at Modane to determine this parameter using a \Co\ source whose two prompt gamma rays can be detected coincidentally by pairs of calorimeter blocks.
An algorithm has been developed in order to characterise the individual time uncertainties, standing at $570\pm130$~ps for the first preliminary result.
This value would have to be updated using the data acquisition planned at Modane by the end of October $2020$, that will be taken now the calorimeter is fully equalised in gain and calibrated in energy.
This would be the occasion to perfect the algorithm developed in the framework of this PhD.

During my PhD, I was also given the opportunity to participate in commissioning data collection and analysis, thus characterising the detector's performances.
In particular, by sending electronic pulses through the signal cables of the calorimeter, the condition of each cable and connector could be checked and corrected if necessary.
The time offset induced by the coaxial cables and calorimeter FEBs has been conducted and two databases were made available to the collaboration.
These two preliminary analyses will have to be completed by a more complete study aimed at characterising the entire signal transmission chain, from the calorimeter to the DAQ.

The commissioning of the calorimeter is now almost complete, and the next step in characterising the detector will be to study the performance of the tracker.
This phase will begin by the end of November, allowing energy calibration of the calorimeter using Bismuth sources, and should be completed by the end of $2020$.
