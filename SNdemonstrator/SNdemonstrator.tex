\chapter{The SuperNemo demonstrator}
\label{ch:detector}

\section{The SuperNemo demonstrator}
\subsection{Comparison with Nemo$3$ experiment}
\subsection{Expermimental design}
\subsection{Sources}
\subsection{Tracker}
\subsection{Calorimeter}
\label{subsec:SN_calo}



\subsubsection{Scintillator}





\subsubsection{Photomultiplier}
\label{sec:calorimeter}
\subsection{Calibration systems}
\subsection{Control Monitoring system}
\subsection{Electronics}

\section{The backgroung of SuperNEMO}
\label{sec:SNbkg}
\subsection{Internal background}
\subsection{External background}
\subsection{Background specifications}

\section{The SuperNemo software}
\label{sec:SNsoftware}
\subsection{Simulation}

As described in Sec.~\ref{sec:SNsoftware} of Chapter~\ref{ch:detector}, the SuperNEMO collaboration developed its own simulation, reconstruction and analysis environment.
The Falaise software, specifically designed by and for the SuperNEMO collaboration, holds the \verb!C++! library for the event reconstruction and analysis of simulated and real data.
Especially, it contains the geometry, the detector material, the event data model, the reconstruction algorithms and the data analysis.
Finally, the SNFee software is a tool package for the configuration, control and monitoring of the SuperNEMO front-end electronics.

\subsection{Reconstruction}
