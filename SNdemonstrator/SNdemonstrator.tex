\chapter{The SuperNemo demonstrator}
\label{ch:detector}

\section{The SuperNemo demonstrator}
\subsection{Comparison with Nemo$3$ experiment}
\subsection{Expermimental design}
\subsection{Sources}
\subsection{Tracker}
\subsection{Calorimeter}
\label{subsec:SN_calo}



\subsubsection{Scintillator}





\subsubsection{Photomultiplier}
\label{sec:calorimeter}
\subsection{Calibration systems}
\subsection{Control Monitoring system}
\subsection{Electronics}

\section{The backgroung of SuperNEMO}
\label{sec:SNbkg}
\subsection{Internal background}
\label{subsec:SNbkg_internal}

Trace quantities of naturally-occurring radioactive isotopes can occasionally produce two-electron events and thus can mimic $\beta\beta$-decay events.
The largest contributions come from isotopes of decay chains of $^{238}$U, $^{232}$Th and $^{40}$K, which disintegration occur inside the source foils, as well as inside the tracking volume.

Décire la contamination mesurée des sources, et du radon dans la partie suivante

\subsection{External background}
Radon:\\
Radon is a noble gas which occurs as an indirect decay product of uranium and thorium.
Due to its chemical properties, radon has a long diffusion length in solids, making it difficult to remove.
Radon contaminations inside the tracker volume is a major background to the rare event experiments such as SuperNEMO.
Simulations show that, to achieve the designed sensitivity, the level of radon must not exceed $0.15$ mBq/m$^{3}$ since its decay daughter \Bi, $\Qbb= 3.2$ MeV can mimic a $\zeronu$ event.
Radon concentration measurements inside the demonstrator tracker have been performed by the SuperNEMO collaboration, revealing an activity of $0.15\pm0.02$ mBq/m$^{3}$, through the combination of an anti-radon tent and an air-flushing method.

%%Repris d'un article -> a changer:
They are outgased in the air from the rock walls of the experimental hall and can enter the detector either through tiny gaps between sectors or through gas pipe joints.
The progeny of radon and thoron produces $\gamma$-rays and $\beta$ decays accompanied by internal conversion (IC), Møller or Compton scattering.
\subsection{Background specifications}
\subsection{Measured demonstrator background levels}

\section{Magnetic field}
\label{sec:magnetic_field}

It is, however, not high enough to impact significantly neither the few muons nor the $\alpha$ particles expected to be detected by the tracker.
Due to their much higher momenta, they will instead leave straight tracks in the wire chamber.

\section{The SuperNemo software}
\label{sec:SNsoftware}
\subsection{Simulation}

As described in Sec.~\ref{sec:SNsoftware} of Chapter~\ref{ch:detector}, the SuperNEMO collaboration developed its own simulation, reconstruction and analysis environment.
The Falaise software, specifically designed by and for the SuperNEMO collaboration, holds the \verb!C++! library for the event reconstruction and analysis of simulated and real data.
Especially, it contains the geometry, the detector material, the event data model, the reconstruction algorithms and the data analysis.
Finally, the SNFee software is a tool package for the configuration, control and monitoring of the SuperNEMO front-end electronics.

\subsection{Reconstruction}
