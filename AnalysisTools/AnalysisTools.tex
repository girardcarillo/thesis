\chapter{Analysis tools}

\section{Internal and external probabilities}
\subsection{Internal probability}
\label{subsec:internal_prob}
Internal probability is a mathematical tool used to quantify the probability that two particles (in this study only electrons will be considered) were emitted simultaneously and at the same location in the source foils.
The internal probability is defined from the associated internal $\chi^{2}$
\begin{equation}
  \chi^{2}_{int}=\frac{((t^{exp}_{1} - \frac{L_{1}}{\beta_{1} c}) - (t^{exp}_{2} - \frac{L_{2}}{\beta_{2} c}))^{2}}{\sigma_{tot}^{2}}\,\text{,}
  \label{eq:int_chi2}
\end{equation}
where c is the speed of light, $t^{exp}_{i}$ is the time measured in calorimeters for the particle $i$, $L_{i}$ is the reconstructed track length, and $\beta_{i}$ is defined as $\sqrt{E_{i}(E_{i} + 2m_{e})} / (E_{i} + m_{e})$, $E_{i}$ being the energy of particle $i$ and $m_{e}$ the electron mass.\\
The denominator in eq.~\ref{eq:int_chi2} is the total uncertainty defined as
\begin{equation}
  \sigma_{tot}=\sqrt{\sigma_{t}^{2}+\sigma_{\left(\frac{L}{\beta c}\right)}^{2}+\sigma_{l}^{2}}\,\text{,}
  \label{eq:sigma_tot}
\end{equation}
where the first term $\sigma^{2}_{t}=\sum_{i=1,2}\sigma^{2}_{t_{i}}$ is the uncertainty the time measurement in the calorimeters with $\sigma_{t}$ defined as
\begin{equation}
  \sigma_{t}=\sqrt{\dfrac{\tau_{\text{SC}}^{2}+\left(\dfrac{\text{FWHM}(\text{TTS})}{2\sqrt{2\ln{2}}}\right)^{2}}{\text{N}_\text{PE}}}\,\text{.}
  \label{eq:sigma_t}
\end{equation}
The second term $\sigma^{2}_{\left(\frac{L}{\beta c}\right)}=\sum_{i=1,2}\sigma^{2}_{\left(\frac{L_{i}}{\beta_{i}c}\right)}$ is the uncertainty depending on the energy of electron $i$ as
\begin{equation}
\sigma_{\left(\frac{L}{\beta c}\right)} = \biggl\lvert \dfrac{\partial t_{th}}{\partial E}  \biggr\rvert \Delta E\,\text{.}
  \label{eq:sigma_L}
\end{equation}
The third term $\sigma_{L}$ represents the typical uncertainty due to track reconstructions of particles\footnote{This value is set to $\sigma_{L}=0.03$ ns in a first approximation (see chapter -ref-).}.\\
Then the internal probability evaluate the difference between experimentally measured time and theoretical times.




\section{Simulations}
\subsection{Modifications of simulation software}
\subsection{Internal background simulations}
\subsection{$\zeronu$ simulations}
