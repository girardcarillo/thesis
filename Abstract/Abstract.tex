%%\chapter*{}



\begin{mdframed}[linecolor=Prune,linewidth=1]
\vspace{-.25cm}
\begin{footnotesize}
\paragraph*{Titre:} Étude de l'influence de la réjection du bruit de fond \Tl\ sur la sensibilité à la décroissance double-bêta sans emission de neutrino, caractérisation des performances en temps du calorimètre du démonstrateur SuperNEMO.

\vspace{-.25cm}
\paragraph*{Mots clés:} Neutrino, Double désintégration bêta, SuperNEMO, Simulation Monte-Carlo, programmation et développement, mise en route du détecteur, réjection de bruit de fond, performances temporelles.

\vspace{-.5cm}
\begin{multicols}{2}
\paragraph*{Résumé:} La physique du neutrino est une des portes possibles pour aller au-delà du Modèle Standard (MS).
En particulier, cette particule est décrite avec une masse nulle par le Lagrangien du MS.
L'étude du mécanisme qui est à l'origine de la génération de leur masse n'est pas connu et dépend de leur nature, que le neutrino soit de Dirac (particule et antiparticule sont différentes) ou de Majorana (le neutrino est son propre antineutrino).

Les expériences NEMO font partie des expériences actuelles qui cherchent à mettre en évidence cette nature, avec une technologie unique alliant reconstruction de trace dans un trajectographe et mesure des énergies et temps de vol dans un calorimètre.
La dernière génération de ce projet est le détecteur SuperNEMO, dont le premier des 20 modules, faisant office de démonstrateur, est en cours d'assemblage au Laboratoire Souterrain de Modane.

Le présent manuscrit décrit le travail de thèse effectué dans cette expérience.
Après avoir rappelé certaines notions liées à la physique du Modèle Standard et au-delà, notamment concernant la physique des neutrinos, le manuscrit présente le démonstrateur SuperNEMO en détail.
Le travail de cette thèse est ensuite décrit dans $4$ chapitres d'analyse.

La sensibilité du démonstrateur à la décroissance $\zeronu$ est étudiée dans différentes conditions de champ magnétique, qui est délivré dans le trajectographe au moyen d'une bobine.
L'influence de la contamination des sources en isotopes naturels est également étudiée.
Il est montré que des coupures sur les données, en particulier dans le canal de détection deux électrons, peuvent améliorer le résultat final de la sensibilité.
Pour les sources \Se, la sensibilité du détecteur final est trouvée à $\Tbeta~>~5,4\times~10^{25}$~années, correspondant à $\langle\mbb\rangle~<~[0,079-0,15]$~eV.
Pour des source \Nd\ $\Tbeta~>~2.4\times~10^{25}$~années serait atteint.
Cela correspond à $\langle\mbb\rangle~<~[0.046-0.15]$~eV, ce qui est meilleur que pour les sources \Se, grâce au meilleur facteur de phase.

Le bruit de fond interne le plus dangereux reste le \Tl, dont l'activité est mesurée comme étant supérieure aux spécifications.
Deux techniques améliorées de réjection de ce fond sont développées, en utilisant notamment le temps de vol mesuré par le calorimètre, et son impact sur la sensibilité de l'expérience est discuté.
Une amélioration de la sensibilité de $6$\% est obtenue en tenant compte des performances raisonnables du calorimètre en matière de temps de vol.

Une description détaillée de la mise en service du calorimètre est donnée, auquel j'ai activement participé pendant mon doctorat.
En particulier, le travail effectué pour vérifier le fonctionnement du calorimètre et de ses câbles de signal est décrit.
La longueur de chaque câble a été mesurée avec précision à l'aide d'une méthode de réflectométrie.
Cela permet d'estimer les retards des signaux, qui ont un impact sur la résolution temporelle évoquée ci-dessus.

Une étude finale visant à déterminer la résolution en temps des modules optiques du calorimètre a été menée, ce qui est crucial pour comprendre et rejeter le bruit de fond de l'expérience.
L'utilisation d'une source \Co\ pour caractériser le calorimètre complet est une idée originale développée dans le cadre de cette thèse, avec la prise en charge à la fois du dispositif expérimental et du développement de l'analyse.
Une caractérisation d'une grande partie du calorimètre a été réalisée, ce qui ouvre la voie à l'étalonnage complet du détecteur avec cette méthode.
En moyenne, la résolution en temps des modules optiques est $570\pm 130$~ps.

\end{multicols}
\end{footnotesize}
\end{mdframed}

\begin{mdframed}[linecolor=Prune,linewidth=1]
\vspace{-.25cm}
\begin{footnotesize}
\paragraph*{Title:} Study of \Tl\ background rejection influence on the neutrinoless double beta decay sensitivity, characterisation of SuperNEMO demonstrator calorimeter timing performance.

\vspace{-.25cm}
\paragraph*{Keywords:} Neutrino, Double beta decay, SuperNEMO, Monte-Carlo simulation, software development, commissioning, background rejection, timing performances.

\vspace{-.5cm}
\begin{multicols}{2}
\paragraph*{Abstract:} The physics of the neutrino is one of the possible doors to go beyond the Standard Model (SM).
In particular, they are described with zero mass by the Lagrangian of the SM.
The study of the mechanism which is at the origin of the generation of their mass is not known and depends on their nature, whether the neutrino is of Dirac (particle and antiparticle are different) or Majorana (the neutrino is its own antineutrino).

The NEMO experiments are part of the current experiments that seek to highlight this nature, with a unique technology combining trace reconstruction in a tracking detector and measurement of energies and times of flight in a calorimeter.
The latest generation of this project is the SuperNEMO detector, of which the first of 20 modules, acting as a demonstrator, is currently being assembled at the Modane Underground Laboratory.

This manuscript describes the PhD work carried out in this experiment.
After recalling certain notions related to the physics of the Standard Model and beyond, notably concerning neutrino physics, the manuscript presents the SuperNEMO demonstrator in detail.
The work of this PhD is then described in $4$ analysis chapters.

The sensitivity of the demonstrator to $\zeronu$ decay is studied under different magnetic field conditions, which is delivered into the tracker by means of a coil.
The influence of the contamination of sources by natural isotopes is also studied.
It is shown that cuts in the data, especially in the two electron detection channel, can improve the final sensitivity result.
For \Se\ sources, the final sensitivity is $\Tbeta~>~5.4\times~10^{25}$~years corresponding to $\langle\mbb\rangle~<~[0.079-0.15]$~eV.
For \Nd\ sources $\Tbeta~>~2.4\times~10^{25}$~years would be reached.
This corresponds to $\langle\mbb\rangle~<~[0.046-0.15]$~eV, better than for \Se\ sources, thanks to its higher phase-space factor.

The most dangerous internal background remains \Tl, whose activity is measured to be higher than the specifications.
Two improved rejection techniques of this background is developed, using in particular the time-of-flight measured with the calorimeter, and its impact on the experiment's sensitivity is discussed.
An improvement of the sensitivity of $6$\% is obtained considering reasonnable calorimeter timing performance.

A detailed description of the commissioning of the calorimeter is given, in which I had an important role during my PhD.
In particular, the work done to verify the operation of the calorimeter and its signal cables is described.
The length of each cable has been accurately measured with a reflectometry method.
This allows to estimate the signal delays, which have an impact on the time resolution discussed above.

A final study to determine the time resolution of the optical modules of the calorimeter was conducted, which is crucial for understanding and rejecting the background of the experiment.
The use of a \Co\ source to characterise the full calorimeter is an original idea developed in the context of this thesis, with the handling of both the experiment setup and the analysis framework.
A characterisation of a large part of the calorimeter has been performed, which paves the way for the full detector calibration with this method.
On average, the time uncertainty stands at $570\pm 130$~ps.


\end{multicols}
\end{footnotesize}
\end{mdframed}

%************************************
\vspace{2cm} % ALIGNER EN BAS DE PAGE
%************************************
\fontfamily{fvs}\fontseries{m}\selectfont
\begin{tabular}{p{14cm}r}
\multirow{3}{16cm}[+0mm]{{\color{Prune} Université Paris-Saclay\\
Espace Technologique / Immeuble Discovery\\
Route de l’Orme aux Merisiers RD 128 / 91190 Saint-Aubin, France}} %% & \multirow{3}{2.19cm}[+9mm]{\includegraphics[height=2.19cm]{e.pdf}}
\\
\end{tabular}







%% \begin{footnotesize}
%% \thispagestyle{empty}
%% \paragraph{Titre :} Étude de l'influence de la réjection du bruit de fond \Tl\ sur la sensibilité à la décroissance double-bêta sans emission de neutrino, caractérisation des performances en temps du calorimètre du démonstrateur SuperNEMO.
%% \paragraph{Mots-clés :} Neutrino, Double désintégration bêta, SuperNEMO, Simulation Monte-Carlo, programmation et développement, mise en route du détecteur, réjection de bruit de fond, performances temporelles.
%% \paragraph{Résumé :} La physique du neutrino est une des portes possibles pour aller au-delà du Modèle Standard (MS).
%% En particulier, cette particule est décrite avec une masse nulle par le Lagrangien du MS.
%% L'étude du mécanisme qui est à l'origine de la génération de leur masse n'est pas connu et dépend de leur nature, que le neutrino soit de Dirac (particule et antiparticule sont différentes) ou de Majorana (le neutrino est son propre antineutrino).

%% Les expériences NEMO font partie des expériences actuelles qui cherchent à mettre en évidence cette nature, avec une technologie unique alliant reconstruction de trace dans un trajectographe et mesure des énergies et temps de vol dans un calorimètre.
%% La dernière génération de ce projet est le détecteur SuperNEMO, dont le premier des 20 modules, faisant office de démonstrateur, est en cours d'assemblage au Laboratoire Souterrain de Modane.

%% Le présent manuscrit décrit le travail de thèse effectué dans cette expérience.
%% Après avoir rappelé certaines notions liées à la physique du Modèle Standard et au-delà, notamment concernant la physique des neutrinos, le manuscrit présente le démonstrateur SuperNEMO en détail.
%% Le travail de cette thèse est ensuite décrit dans $4$ chapitres d'analyse.

%% La sensibilité du démonstrateur à la décroissance $\zeronu$ est étudiée dans différentes conditions de champ magnétique, qui est délivré dans le trajectographe au moyen d'une bobine.
%% L'influence de la contamination des sources en isotopes naturels est également étudiée.
%% Il est montré que des coupures sur les données, en particulier dans le canal de détection deux électrons, peuvent améliorer le résultat final de la sensibilité.
%% Pour les sources \Se, la sensibilité du détecteur final est trouvée à $\Tbeta~>~5,4\times~10^{25}$~années, correspondant à $\langle\mbb\rangle~<~[0,079-0,15]$~eV.
%% Pour des source \Nd\ $\Tbeta~>~2.4\times~10^{25}$~années serait atteint.
%% Cela correspond à $\langle\mbb\rangle~<~[0.046-0.15]$~eV, ce qui est meilleur que pour les sources \Se, grâce au meilleur facteur de phase.

%% Le bruit de fond interne le plus dangereux reste le \Tl, dont l'activité est mesurée comme étant supérieure aux spécifications.
%% Deux techniques améliorées de réjection de ce fond sont développées, en utilisant notamment le temps de vol mesuré par le calorimètre, et son impact sur la sensibilité de l'expérience est discuté.
%% Une amélioration de la sensibilité de $6$\% est obtenue en tenant compte des performances raisonnables du calorimètre en matière de temps de vol.

%% Une description détaillée de la mise en service du calorimètre est donnée, auquel j'ai activement participé pendant mon doctorat.
%% En particulier, le travail effectué pour vérifier le fonctionnement du calorimètre et de ses câbles de signal est décrit.
%% La longueur de chaque câble a été mesurée avec précision à l'aide d'une méthode de réflectométrie.
%% Cela permet d'estimer les retards des signaux, qui ont un impact sur la résolution temporelle évoquée ci-dessus.

%% Une étude finale visant à déterminer la résolution en temps des modules optiques du calorimètre a été menée, ce qui est crucial pour comprendre et rejeter le bruit de fond de l'expérience.
%% L'utilisation d'une source \Co\ pour caractériser le calorimètre complet est une idée originale développée dans le cadre de cette thèse, avec la prise en charge à la fois du dispositif expérimental et du développement de l'analyse.
%% Une caractérisation d'une grande partie du calorimètre a été réalisée, ce qui ouvre la voie à l'étalonnage complet du détecteur avec cette méthode.
%% En moyenne, la résolution en temps des modules optiques est $570\pm 130$~ps.

%% \newpage
%% \thispagestyle{empty}

%% \paragraph{Title:} Study of \Tl\ background rejection influence on the neutrinoless double beta decay sensitivity, characterisation of SuperNEMO demonstrator calorimeter timing performance.
%% \paragraph{Keywords:} Neutrino, Double beta decay, SuperNEMO, Monte-Carlo simulation, software development, commissioning, background rejection, timing performances.
%% \paragraph{Abstract:} The physics of the neutrino is one of the possible doors to go beyond the Standard Model (SM).
%% In particular, they are described with zero mass by the Lagrangian of the SM.
%% The study of the mechanism which is at the origin of the generation of their mass is not known and depends on their nature, whether the neutrino is of Dirac (particle and antiparticle are different) or Majorana (the neutrino is its own antineutrino).

%% The NEMO experiments are part of the current experiments that seek to highlight this nature, with a unique technology combining trace reconstruction in a tracking detector and measurement of energies and times of flight in a calorimeter.
%% The latest generation of this project is the SuperNEMO detector, of which the first of 20 modules, acting as a demonstrator, is currently being assembled at the Modane Underground Laboratory.

%% This manuscript describes the PhD work carried out in this experiment.
%% After recalling certain notions related to the physics of the Standard Model and beyond, notably concerning neutrino physics, the manuscript presents the SuperNEMO demonstrator in detail.
%% The work of this PhD is then described in $4$ analysis chapters.

%% The sensitivity of the demonstrator to $\zeronu$ decay is studied under different magnetic field conditions, which is delivered into the tracker by means of a coil.
%% The influence of the contamination of sources by natural isotopes is also studied.
%% It is shown that cuts in the data, especially in the two electron detection channel, can improve the final sensitivity result.
%% For \Se\ sources, the final sensitivity is $\Tbeta~>~5.4\times~10^{25}$~years corresponding to $\langle\mbb\rangle~<~[0.079-0.15]$~eV.
%% For \Nd\ sources $\Tbeta~>~2.4\times~10^{25}$~years would be reached.
%% This corresponds to $\langle\mbb\rangle~<~[0.046-0.15]$~eV, better than for \Se\ sources, thanks to its higher phase-space factor.

%% The most dangerous internal background remains \Tl, whose activity is measured to be higher than the specifications.
%% Two improved rejection techniques of this background is developed, using in particular the time-of-flight measured with the calorimeter, and its impact on the experiment's sensitivity is discussed.
%% An improvement of the sensitivity of $6$\% is obtained considering reasonnable calorimeter timing performance.

%% A detailed description of the commissioning of the calorimeter is given, in which I had an important role during my PhD.
%% In particular, the work done to verify the operation of the calorimeter and its signal cables is described.
%% The length of each cable has been accurately measured with a reflectometry method.
%% This allows to estimate the signal delays, which have an impact on the time resolution discussed above.

%% A final study to determine the time resolution of the optical modules of the calorimeter was conducted, which is crucial for understanding and rejecting the background of the experiment.
%% The use of a \Co\ source to characterise the full calorimeter is an original idea developed in the context of this thesis, with the handling of both the experiment setup and the analysis framework.
%% A characterisation of a large part of the calorimeter has been performed, which paves the way for the full detector calibration with this method.
%% On average, the time uncertainty stands at $570\pm 130$~ps.


%% \end{footnotesize}
